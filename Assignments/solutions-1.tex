\documentclass{article}
\usepackage{amsmath, amssymb, amsfonts, amsthm, mathtools}
\usepackage[utf8]{inputenc}
\usepackage[inline]{enumitem}
\usepackage{cancel}
\usepackage{soul}
\usepackage{hyperref}
\usepackage{centernot}
\usepackage{pifont}
\usepackage{fancyvrb}
\newtheorem{theorem}{Theorem}
\setlength\parindent{0pt}
\let\emptyset\varnothing

\usepackage{xcolor}
\definecolor{mybgcolor}{RGB}{50, 50, 50} %46, 51, 63

\usepackage{pagecolor}
\pagecolor{mybgcolor}
\color{white}

\usepackage{titlesec}
\titleformat{\section}[block]
  {\normalfont\scshape}{\S\thesection}{0.25cm}{\large}

\usepackage{geometry}
\geometry{
	a4paper,
	total={170mm,257mm},
	left=20mm,
	top=20mm,
}

\title{Solutions and Marking Scheme\\Assignment 1}	% change
\author{Aryaman Maithani}
\date{ }

\begin{document}
\maketitle

\hrulefill

\section{About the Marking Scheme}
Grading and giving marks is often one of the necessary evils one must face while conducting a course. However, in this case, they can be avoided and that is precisely what I shall do.\\
Thus, the assignment is of exactly zero (0) marks and that is what everyone (who submitted) will get. I did have half a mind of deducting marks for those who couldn't name the files properly despite me explicitly stating it but oh, well.\\\\
Now, you might get disappointed if you felt that you did extremely well and needed a high score to validate that. In which case, my apologies.\\
You might also get into a philosophical (or mathematical) crisis of what it \emph{means} to get $0/0.$ In which case, I request you to not think about it and move on to the next section.

\hrulefill

\section{Solutions}
\begin{enumerate}[label = Q.\arabic*.] 
	\setcounter{enumi}{-1}
	\item Let $S \subset \mathbb{R}^2.$ State what it means for $S$ to be star-shaped.\\\\
	\textbf{Solution.}\\
	$S$ is said to be star-shaped if there exists a point $p_0 = (x_0, y_0) \in S$ such that for every point $p = (x, y) \in S,$ the line segment joining $p$ and $p_0$ lies in $S.$\\
	More explicitly, $(1 - t)p_0 + tp \in S$ for all $t \in [0, 1].$
	\item Show that $\mathbb{R}^2\setminus\{(0, 0)\}$ is not star shaped.\\
	\emph{Bonus.} Let $S$ be any finite subset of $\mathbb{R}^2.$ \\
	Prove or disprove that $\mathbb{R}^2\setminus S$ is star-shaped.\\\\
	\textbf{Solution.}\\
	Let $D = \mathbb{R}^2\setminus\{(0, 0)\}.$\\
	Let us assume, for the sake of contradiction, that $D$ is star-shaped.\\
	Thus, there exists a point $p_0 = (x_0, y_0) \in D$ with the property mentioned above.\\
	Note that it is not possible for $x_0$ and $y_0$ to both be $0.$ Thus, we have that $(-x_0, -y_0) \neq (0, 0)$ and thus, $p = (-x_0, -y_0) \in D.$\\
	By our assumption of $D$ being star-shaped, we get that the line segment joining $p_0$ and $p$ lies in $D.$\\
	In particular, the midpoint obtained by setting $t = 0.5$ in the above definition must lie in $D.$ However, this midpoint is precisely $(0, 0).$ A contradiction. \hfill $\blacksquare$\\\\
	\emph{Bonus.} We first consider the case that $S = \emptyset.$ In this case, we have that $\mathbb{R}^2\setminus S = \mathbb{R}^2,$ which is indeed star-shaped.\\
	Now, let us consider the case that $S \neq \emptyset.$\\
	Then, $S = \{(x_1,y_1),\;(x_2,y_2),\;\ldots,\;(x_n, y_n)\}$ for some $n \ge 1.$ Let $D = \mathbb{R}^2\setminus S.$\\
	We claim that $D$ is not star-shaped. Once again, we proceed by contradiction.\\
	Let us assume that $D$ is indeed star-shaped. Let $p_0 = (x_0, y_0) \in D$ be as before.\\
	Let $p = (x_1, y_1) \in S.$ Consider the ray emanating from $p_0$ passing through $p.$ Now, consider the truncated ray obtained after excluding the line segment joining $p_0$ and $p.$ In other words, consider the following subset of $\mathbb{R}^2:$
	\[R = \{(1 - t)p_0 + tp \mid t > 1\}.\]
	Note that $R$ is infinite and hence, there exists $q \in R$ such that $q \notin S.$ Hence, $q \in D.$\\
	As $p_0, q \in D$ and $D$ is star-shaped (by assumption), we get that the line segment joining $p_0$ and $q$ must lie in $D.$ However, this line segment contains $p \notin D.$ A contradiction.\hfill $\blacksquare$\\\\
	\item Solve Q.10.(i) of Sheet-1 to completion.\\\\
	\textbf{Solution.}\\
	We make the substitution $X = x - h_0$ and $Y = y - k_0$ such that $(h, k) = (h_0, k_0)$ is a solution of
	\begin{align*} 
		-h + 2k - 1 &= 0,\\
		4h - 3k - 6 &= 0.
	\end{align*}
	Solving the above gives us $h_0 = 3$ and $k_0 = 2.$\\
	This gives us the ODE
	\[\dfrac{dY}{dX} = \dfrac{-X + 2Y}{4X - 3Y}.\]
	To solve this, we make the substitution $Y = vX$ which gives us $dY = vdX + Xdv.$ Thus, the ODE changes to
	\begin{align*} 
		& v + X\dfrac{dv}{dX} = \dfrac{2v - 1}{-3v + 4}\\
		\implies & X\dfrac{dv}{dX} = \dfrac{3v^2 -2v - 1}{-3v + 4}\\
		\implies & \dfrac{-3v + 4}{3v^2 - 2v - 1}dv = \dfrac{1}{X}dX\\
		\implies & \left(-\frac{15}{4}\cdot\frac{1}{3v + 1} + \frac{1}{4}\cdot\frac{1}{v - 1}\right)dv = \frac{1}{X}dX\\
		\implies & \int\left(-\frac{15}{4}\cdot\frac{1}{3v + 1} + \frac{1}{4}\cdot\frac{1}{v - 1}\right)dv = \int\frac{1}{X}dX\\
		\implies & -\frac{5}{4}\ln|3v + 1| + \frac{1}{4}\ln|v - 1| = \ln|X| + C\\
		\implies & 4\ln|X| - \ln|v - 1| + 5\ln|3v +  1| = C'\\
		\implies & X^4|(v - 1)^{-1}(3v + 1)^5| = K\\
		\implies & |(vX - X)^{-1}(3vX + X)^5| = K
	\end{align*}
	Recalling that $vX = Y$ gives us $|(Y - X)^{-1}(3Y + X)^5| = K.$\\
	Substituting $X = x - 3$ and $Y = y - 2$ back gives us
	\[|3y + x - 9|^5 = K |y - x + 1|.\]
\end{enumerate}

\section{Some Mistakes}
\begin{enumerate}[label = Q.\arabic*.] 
	\setcounter{enumi}{-1}
	\item 
	\begin{enumerate} 
		\item It is important to say that ``there exists $(x_0, y_0) \in S$'' and not just ``there exists $(x_0, y_0)$.'' It belonging to $S$ is important.
		\item Not a mistake per se but no need to write ``there exists \emph{at least} one point ...''. If you simply write ``there exits a point,'' it is implied that there is \emph{at least} one. We don't assume uniqueness or anything.\\
		However, it is not wrong if you do write.
		\item One more minor nitpicking is that you don't have to explicitly say that $p \neq p_0.$
	\end{enumerate}
	\item 
	\begin{enumerate} 
		\item Note that if $(x_0, y_0) \in D,$ then what you can say is that: $x_0 \neq 0$ \emph{\textbf{or}} $y_0 \neq 0.$\\
		What you can \textbf{not} say is that: $x_0 \neq 0$ \emph{and} $y_0 \neq 0.$\\
		If you do that, then you miss points like $(1, 0)$ or $(0, 1),$ which do indeed belong to $D.$
		\item Make sure you write ``line segment'' instead of ``line''.
		\item Notation: The symbol for set difference is $\setminus$ and not $/.$
	\end{enumerate}
	\emph{Bonus:}
	\begin{enumerate}
		\item Only one has taken care of the case $S = \emptyset.$ Note that your arguments require $S$ being nonempty because you say something like ``let $(a, b) \in S$'' which is not possible unless $S \neq \emptyset.$
		\item For some reason, many have argued in the following manner: ``We may translate the situation so that $S$ contains $(0, 0)$ and then we are done by the previous part.''\\
		This is \textbf{not} a valid argument because you seem to be implying that any subset of $\mathbb{R}^2$ not containing the origin is not star-shaped. This is not true. For example, consider the following subset of $\mathbb{R}^2:$
		\[\mathbb{R}^2 \setminus ((-\infty, 0]\times\{0\}),\]
		that is, $\mathbb{R}^2$ minus the non-positive $x-$axis. This is indeed star-shaped. (Prove!)\\
		However, this does not contain the origin.\\\\
		The fact that $S$ is finite is of crucial importance and you have to use that.
	\end{enumerate}
	\item 
	\begin{enumerate} 
		\item People have messed up whether $X = x + h$ or $-h.$ Note that the calculation done in class was under the substitution $x = X + h.$ If you take $X = x + h,$ then note that you'll get $h = -3.$
		\item Many have made the common mistake of confusing the denominator to be $3v - 1$ even after writing $3v + 1$ in the previous step. Not sure why.
		\item Some have forgotten the $1/3$ that comes from integrating $(3v + 1)^{-1}.$
		\item Don't drop the modulus unnecessarily.
	\end{enumerate}
\end{enumerate}

\section{\LaTeX\ Suggestions}
\begin{enumerate} 
	\item Always use the \verb+$ $+ environment for variables, even if you don't necessarily need a math symbol. For example, if you're referring to the subset $S$ of $\mathbb{R}^2,$ don't write S. Same goes for something like ``Y = y - 3''.
	\item The command \verb+\setminus+ can be used for the $\setminus$ instead of using something else. Even if it looks the same, it's a question of being ``correct.''
	\item Use the command \verb+\left(...\right)+ to make sure the brackets are big enough. This is a better alternative to \verb+\bigg(...\bigg)+.
\end{enumerate}
\end{document}