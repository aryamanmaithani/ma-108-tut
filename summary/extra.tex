
\sec{Some extra results}
This subsection can be skipped by the reader. However, I find this quite interesting and thus, I have kept it here anyway.
\begin{mdframed}[style=boxstyle, frametitle={The sinc integral}]
	In this box, we compute $\displaystyle\int_{0}^{\infty} \dfrac{\sin x}{x} dx.$\\~\\
	It is known that the improper integral cannot be expressed in terms of elementary functions. Thus, it is quite neat (to me) that we can compute the above (definite) integral.\\
	We first define the following function for $t > 0:$
	\begin{equation*} 
		f(t) = \int_{0}^{\infty} \dfrac{\sin tx}{x} dx.
	\end{equation*}
	(It will have to be justified that the integral on the right does indeed exist for all $t > 0$. This is not too tough.)\\
	Taking the Laplace transform on both sides (and interchanging $\int$ and $\mathcal{L}$), we get:
	\begin{align*} 
		(\mathcal{L}f)(s) &= \int_{0}^{\infty} \dfrac{1}{x} \mathcal{L}(\sin tx) dx\\
		&= \int_{0}^{\infty} \dfrac{1}{x}\cdot\dfrac{x}{x^2 + s^2} dx\\
		&= \int_{0}^{\infty} \dfrac{1}{x^2 + s^2} dx\\
		&= \dfrac{\pi}{2s}\\
		&= \left(\mathcal{L}\left(\dfrac{\pi}{2}\right)\right)(s).
	\end{align*}
	Thus, $f(t) = \pi/2,$ identically. (Both are continuous.)\\
	In particular, we have:
	\begin{equation*} 
		f(1) = \int_{0}^{\infty} \dfrac{\sin x}{x} dx = \dfrac{\pi}{2}.
	\end{equation*}
\end{mdframed}
\begin{mdframed}[style=boxstyle, frametitle={Generalised Fresnel integrals}]
	We calculate integrals of the form $\displaystyle\int_{0}^{\infty} \sin(x^a) dx$ for $a > 1.$ The case $a = 2$ is of particular importance in optics or whatever.\\
	Let $a > 1$ be given. For $t > 0,$ we define
	\begin{equation*} 
		f(t) := \int_{0}^{\infty} \sin tx^a dx.
	\end{equation*}
	Taking Laplace transforms as earlier, we get

	\begin{align*} 
		F(s) &= \int_{0}^{\infty} \dfrac{x^a}{s^2 + x^{2a}} dx\\
		&\text{put } x^{2a} = s^2 u\\
		F(s) &= \int_{0}^{\infty} \dfrac{s\sqrt{u}}{s^2(1 + u)}\cdot\dfrac{s^{1/a}}{2a}\cdot u^{\frac{1}{2a} - 1} du\\
		&= \dfrac{1}{s^{1 - \frac{1}{a}}}\cdot\dfrac{1}{2a}\int_{0}^{\infty} \dfrac{u^{\frac{1}{2} + \frac{1}{2a} - 1}}{1 + u} du\\
		&= \dfrac{1}{s^{1 - \frac{1}{a}}}\dfrac{1}{2a}\pi\operatorname{cosec}\left(\dfrac{\pi}{2} + \dfrac{\pi}{2a}\right) \qquad (\text{Theorem \ref{thm:randomintegral}.})\\
		&= \dfrac{\pi}{2a}\dfrac{1}{s^{1 - \frac{1}{a}}}\operatorname{sec}\left(\dfrac{\pi}{2a}\right).
	\end{align*}
	Recall that 
	\begin{equation*} 
		\mathcal{L}(t^k) = \dfrac{\Gamma(k+1)}{s^{k+1}}.
	\end{equation*}
	Thus, we see that
	\begin{equation*} 
		f(t) = \dfrac{\pi}{2a}\operatorname{sec}\left(\dfrac{\pi}{2a}\right)\dfrac{t^{-1/a}}{\Gamma\left(1 - \frac{1}{a}\right)}.
	\end{equation*}
	In particular (setting $t = 1$), we see that
	\begin{equation*} 
		\int_{0}^{\infty} \sin x^a dx = \dfrac{\pi}{2a}\operatorname{sec}\left(\dfrac{\pi}{2a}\right)\dfrac{1}{\Gamma\left(1 - \frac{1}{a}\right)}.
	\end{equation*}
	Thus, we also see that
	\begin{equation*} 
		\int_{0}^{\infty} \sin x^2 dx = \dfrac{\sqrt{\pi}}{2\sqrt{2}}.
	\end{equation*}
\end{mdframed}
\exercise{Show that $\displaystyle\int_{0}^{\infty} \cos x^2 dx = \dfrac{\pi}{2\sqrt{2}}.$}

\begin{mdframed}[style=boxstyle, frametitle={The Beta-Gamma Relation}]
	Let $a, b > 0.$ We first calculate the convolution $t^a*t^b:$
	\begin{align*} 
		t^a * t^b &= \int_{0}^{t} (t - x)^ax^b dx\\
		& \text{put }u = x/t\\
		&= \int_{0}^{1} (t - ut)^a(ut)^b tdu\\
		&= t^{a + b + 1}\int_{0}^{1} (1 - u)^au^b du\\
		&= t^{a + b + 1}B(a + 1, b + 1).
	\end{align*}
	Thus, $t^{a - 1}*t^{b - 1} = t^{a + b - 1}B(a, b).$\\
	Taking the Laplace transform on both sides gives us:
	\begin{align*} 
		\mathcal{L}(t^{a - 1}*t^{b - 1}) &= B(a, b)\mathcal{L}(t^{a + b - 1})\\
		\implies \mathcal{L}(t^{a - 1})\mathcal{L}(t^{b - 1}) &= B(a, b)\dfrac{\Gamma(a + b)}{s^{a + b}}\\
		\implies \dfrac{\Gamma(a)}{s^{a}}\dfrac{\Gamma(b)}{s^b} &= B(a, b)\dfrac{\Gamma(a + b)}{s^{a + b}}\\
		\implies \Gamma(a)\Gamma(b) &= \Gamma(a + b)B(a, b).
	\end{align*}
\end{mdframed}

\begin{mdframed}[style=boxstyle, frametitle={A little integral}]
	Before the next nice result, let us prove the following result:
	\begin{equation} \label{eq:logint}
		\ln N = \int_{0}^{\infty} \dfrac{1}{t}(e^{-t} - e^{-Nt}) dx.
	\end{equation}
	(That the above exists for all $N > 0$ can be checked.)\\
	\begin{align*} 
		\int_{0}^{\infty} \dfrac{1}{t}(e^{-t} - e^{-Nt}) dx &= \int_{0}^{\infty} \int_{1}^{N} e^{-tx} dx dt\\
		&= \int_{1}^{N} \int_{0}^{\infty} e^{-tx} dt dx\\
		&= \int_{1}^{N} \dfrac{1}{x} dx\\
		&= \ln N.
	\end{align*}
\end{mdframed}

\begin{mdframed}[style=boxstyle, frametitle={Another pesky integral}]
	The nice result is almost here. However, first let's get this out of the way
	\begin{equation} \label{eq:pesky}
		\int_{0}^{\infty} \left(\dfrac{1}{1 - e^{-t}} - \dfrac{1}{t}\right)e^{-t} dt = -\int_{0}^{\infty} e^{-t}\ln t dt
	\end{equation}
	Note that
	\begin{align*} 
		&\int_{0}^{\infty} \left(\dfrac{1}{1 - e^{-t}} - \dfrac{1}{t}\right)e^{-t} dt \\
		&= \int_{0}^{\infty} e^{-t}\dfrac{d}{dt}\left(\ln\left(\dfrac{e^t - 1}{t}\right)\right) dt \quad \text{(Just compute)}\\
		&= e^{-t}\ln\left(\dfrac{e^t - 1}{t}\right)\Bigg|_0^\infty + \int_{0}^{\infty} e^{-t}\ln\left(\dfrac{e^t - 1}{t}\right) dt \quad \text{(by parts)}\\
		&= 0 + \int_{0}^{\infty} e^{-t}\ln\left(\dfrac{e^t - 1}{t}\right) dt\\
		&= \int_{0}^{\infty} e^{-t}\ln(e^t - 1)dt - \int_{0}^{\infty} e^{-t}\ln t dt.
	\end{align*}
	Thus, it suffices to show that the first integral is zero. Call it $I.$ \\
	We have
	\begin{align*} 
		I &= \int_{0}^{\infty} e^{-t}\ln(e^t - 1)dt\\
		& \text{put } u =\ln(e^t - 1)\\
		&= \int_{-\infty}^{\infty} \dfrac{ue^u}{(1 + e^u)^2} du\\
		&= 0 \quad \text{(the integrand is odd)}
	\end{align*}
	Thus, we have established the result.
\end{mdframed}

\newpage

\begin{mdframed}[style=boxstyle, frametitle={Smol gamma and big Gamma}]
	The Euler Mascheroni constant $\gamma$ is defined as
	\begin{equation*} 
		\gamma := \lim_{N\to \infty}\left(1 + \dfrac{1}{2} + \cdots + \dfrac{1}{N} - \ln N\right).
	\end{equation*}
	Note that $\displaystyle\int_{0}^{\infty} e^{-kt} dt = \dfrac{1}{k}.$\\~\\
	Thus,
	\begin{equation*}
		1 + \dfrac{1}{2} + \cdots + \dfrac{1}{N} = \int_{0}^{\infty} \sum_{k=1}^{N} e^{-kt} dt.
	\end{equation*}
	Now, using (\ref{eq:logint}), we get the following:
	\begin{equation*} 
		1 + \dfrac{1}{2} + \cdots + \dfrac{1}{N} - \ln N = \int_{0}^{\infty} \left(\sum_{k=1}^{N} e^{-kt} - \dfrac{1}{t}(e^{-t} - e^{-Nt})\right) dt
	\end{equation*}
	Letting $N \to \infty,$ we get
	\begin{align*} 
		\lim_{N\to \infty}\left(1 + \dfrac{1}{2} + \cdots + \dfrac{1}{N} - \ln N\right) &= \int_{0}^{\infty} \left(\dfrac{1}{1 - e^{-t}} - \dfrac{1}{t}\right)e^{-t} dt\\
		&= -\int_{0}^{\infty} e^{-t}\ln t dt.
	\end{align*}
	We used (\ref{eq:pesky}) for the last equality.\\
	Thus, we have another expression for $\gamma,$ namely
	\begin{equation*} 
		\boxed{\gamma = -\int_{0}^{\infty} e^{-t}\ln t dt.}
	\end{equation*}

	Now, recalling big $\Gamma,$ note that

	\begin{align*} 
		\Gamma(x) &= \int_{0}^{\infty} e^{-t}t^{x - 1} dt\\
		\implies \Gamma'(x) &= \int_{0}^{\infty} e^{-t}t^{x - 1}\ln t dt\\
		\implies \Gamma'(1) &= \int_{0}^{\infty} e^{-t}\ln t dt\\
		&= -\gamma.
	\end{align*}

	Thus, we also have

	\begin{equation*} 
		\boxed{\gamma = -\Gamma'(1).}
	\end{equation*}
\end{mdframed}