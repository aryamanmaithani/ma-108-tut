\sec{Picard's Iterates}
\subsection{Introduction}
\begin{mdframed}[style=boxstyle, frametitle={The Setup}]
	Consider the initial value problem (IVP)
	\[y' = f(t, y); \;\; y(a) = b.\]
	Corresponding to this, we set up the following \emph{integral equation}
	\[\phi(t) = b + \int_{a}^{t} f(s, \phi(s)) ds.\]
	It can be verified a solution $\phi_0$ to the integral equation is a solution of the original DE as well. (And vice-versa.)\\
\end{mdframed}
Now, we describe a method to solve the integral equation.

\newpage

\begin{mdframed}[style=boxstyle, frametitle={Picard's Iteration Method}]
	We recursively define a family of functions $\phi_n(t)$ for every $n \ge 0$ as follows:
	\begin{align*} 
		\phi_0 &\equiv b,\\
		\phi_{n + 1}(t) &= b + \int_{a}^{t} f(s, \phi_n(s)) ds \quad \text{for }n \ge 0.
	\end{align*}
\end{mdframed}
Under suitable conditions, the sequence of functions $\left(\phi_n\right)$ converges to a function
\[\phi(t) = \lim_{n\to \infty}\phi_n(t),\]
which is a solution to the IVP.

\exercise{%
Solve the IVP:
\[y'(t) = 2t(1 + y); \;\; y(0) = 0.\]
}